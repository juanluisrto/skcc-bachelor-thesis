\newpage
\thispagestyle{empty}
~\\
\vfill
{ \setstretch{1.1}
\subsection*{Author}
Juan Luis Ruiz-Tagle <jlrto@kth.se>\\
Information and Communication Technology\\
KTH Royal Institute of Technology

\subsection*{Place for Project}
Stockholm, Sweden\\

\subsection*{Examiner}
Anne Håkansson
Software and Computer Systems ICT\\
KTH Royal Institute of Technology

\subsection*{Supervisor }
Oskar Henriksson\\
Head of Recruitment\\
Slagkryssaren
~

}

\newpage
\section*{Abstract}
\vspace{2cm}
TensorFlow is an open-source library developed by Google for machine learning applications. It can be run in different environments such as desktop computers, servers or even browsers. For running TensorFlow on mobile devices there exist TensorFlow Mobile and the recently launched TensorFlow Lite. The latter is an improved version of the former, which claims to have several performance advantages over its predecessor and is meant to substitute it on the long run. This thesis presents an analysis of how the execution speed of a convolutional neural network varies depending on if it is being run in TensorFlow Lite or in TensorFlow Mobile. Such convolutional neural network is trained to analyze pictures of credit cards. In the thesis work the project is delimited by a set of requirements coming from the stakeholders and a system architecture which satisfies them is presented. The implementation of such system in the form of an application is described and the whole process is finally evaluated. This application is programmed to run the same network with both versions of TensorFlow and is installed on several devices to measure their performance by running some tests. The collected empirical data, which is presented and analysed, shows that TensorFlow Lite's performance is still lower than its counterpart's when running convolutional neural networks. This is due to the early stage of development of the library, which is not optimized enough yet. 

\subsection*{Keywords}
Machine Learning, TensorFlow Lite, Convolutional neural networks, Performance

\newpage
\section*{Abstract}
TensorFlow är en open-source programbibliotek utvecklat av Google för maskininlärningsapplikationer. Det kan användas i olika miljö såsom datorer, servrar eller webbläsare till och med. För att köra TensorFlow på mobilenheter finns det TensorFlow Mobile och den nyligen utsläppt TensorFlow Lite version som är en förbättrad version av den andra. Det påstås att ha flera prestationfördelar över dess företrädare och är menat att ersätta det. Denna uppsats presenterar en analys av hur exekveringshastighet av en convolutional neural network växlar beroende på om det körs med TensorFlow Lite eller TensorFlow Mobile. Sådant neuronät är tränat för att analysera kreditkort bilder. Examensarbetet är gränsat av några förutsättningar som kommer från delägarna och en systemarkitektur som motsvarar dem föreställes. Implementeringen av sådant system i form av en app beskrivs och hela processen värderas. Denna app programmerades för att köra samma nät med bägge TensorFlow versioner och installerades i flera enheter för att mäta dess prestation genom att utföra några tester. Den samlade datan, som presenteras, bevisar att TensorFlow Lites prestanda ligger fortfarande lägre än TensorFLow Mobile när de kör convolutional neural networks. Detta sker på grund av den tidiga utvecklingsståndet av programbiblioteket, som inte är optimerat ännu.

\subsection*{Nyckelord}
Maskininlärning, TensorFlow Lite, Neuronät, Prestation


%% The table of content 
\etocdepthtag.toc{mtchapter}
\etocsettagdepth{mtchapter}{subsection}
\etocsettagdepth{mtappendix}{none}
\newpage
\thispagestyle{fancy}
\tableofcontents
\thispagestyle{fancy} % Fixing a bug where the page number would randomly fail to be right justified.

